%
%
\documentclass{article}
\usepackage{amsmath}
\usepackage{graphicx}
\usepackage{caption}
\usepackage{mathrsfs}
\usepackage{eufrak}
\usepackage{amsfonts}
\usepackage[margin=4cm]{geometry}
\newcommand{\cE}{\mathcal{E}}                
\newcommand{\fP}{\mathfrak{P}}               
\newcommand{\fH}{\mathfrak{H}}               
\newcommand{\fG}{\mathfrak{G}}               
\newcommand{\fK}{\mathfrak{K}}               
\newcommand{\sP}{\mathscr{P}}                
\newcommand{\sH}{\mathscr{H}}                
\newcommand{\sG}{\mathscr{G}}                
\newcommand{\sK}{\mathscr{K}}                

\begin{document}

\title{Rescaling the equations - Summary}
\author{Dominic Skinner}
\maketitle
Here we will recap the scalings, and then explain how we will switch between
the ``$h''\to1$'' and the ``$\lambda = 1$'' scalings.
\subsection*{Full equations:}
%
\begin{equation}
 \left( \begin{array}{c} p(z) \\ 0 \end{array} \right) =
\frac{E}{4\pi (1-\nu^2)} \int_0^{\infty} 
\underline{\underline{K}}(x-z) 
\left( \begin{array}{c} g'(x) \\ h'(x) \end{array} \right) dx
\end{equation}
%
\begin{equation}
12\mu c = h^2 p'
\end{equation}
%
\begin{equation}
\left\{ \begin{array}{ccc}
\displaystyle \lim_{x\to\infty} h''(x) & = & \frac{12(1-\nu^2)}{E\ell^3} M \\
\displaystyle \lim_{x\to\infty} g'(x) & = & \frac{6(1-\nu^2)}{E\ell^3} M 
\end{array} \right.
\end{equation}
%
\begin{equation}
K_I = \lim_{x\to 0} \frac{E}{1-\nu^2} \sqrt{\frac{\pi}{8}} \sqrt{x} h'(x)
\end{equation}
%
\section{Scale $h''\to1$}
The more common used scaling.
\begin{alignat*}{2}
x &= \ell \xi  & p(x) &= \beta_1 \sP (\xi) \\
h(x) &= \alpha_1 \sH(\xi) \qquad & \qquad g(x) &= \alpha_1 \sG(\xi) \\
\underline{\underline{K}}(x) &= \frac{1}{\ell} 
\underline{\underline{\Lambda}}(\xi) & 
K_I &= M \ell^{-3/2} \sK 
\end{alignat*}
Where  $\displaystyle \alpha_1 = \frac{12(1-\nu^2)M}{E\ell}$ and  
$\displaystyle \beta_1 = \frac{3M}{\pi \ell^2}$.
The equations become
\\
\begin{equation}
 \left( \begin{array}{c} \sP(\xi) \\ 0 \end{array} \right) =
\int_0^{\infty} 
\underline{\underline{\Lambda}}(\tilde{\xi}- \xi) 
\left( \begin{array}{c} \sG'(\tilde{\xi}) \\[4pt] \sH'(\tilde{\xi}) \end{array} 
\right) d\tilde{\xi}
\end{equation}
%
\begin{equation}
\lambda  = \sH^2 \sP'
\end{equation}
%
\begin{equation}
\left\{ \begin{array}{ccc}
\lim_{\xi\to\infty} \sH''(x) & = & 1 \\[4pt]
\lim_{\xi\to\infty} \sG'(x) & = & 1/2  
\end{array} \right.
\end{equation}
%
\begin{equation}
\sK = \lim_{\xi\to 0} 3 \sqrt{2\pi} \sqrt{\xi} \sH'(\xi)
\end{equation}
\section{Scale $\lambda=1$}
%
\begin{alignat*}{2}
x &= \ell \xi  & p(x) &= \beta_2 \fP (\xi) \\
h(x) &= \alpha_2 \fH(\xi) \qquad & \qquad g(x) &= \alpha_2 \fG (\xi) \\
\underline{\underline{K}}(x) &= \frac{1}{\ell} 
\underline{\underline{\Lambda}}(\xi) & 
K_I &= M \ell^{-3/2} \fK 
\end{alignat*}
%
\\
Where $\displaystyle \alpha_2 = \left(\frac{48\pi(1-\nu^2) \mu c \ell^2}{E}
\right)^{1/3}$, and $\displaystyle \beta_2 = \left(\frac{3\mu c E^2}{4\pi^2 
(1 - \nu^2)^2 \ell}\right)^{1/3}$.
We then have the relevant equations as 
\\
\begin{equation}
 \left( \begin{array}{c} \fP(\xi) \\ 0 \end{array} \right) =
\int_0^{\infty} 
\underline{\underline{\Lambda}}(\tilde{\xi}- \xi) 
\left( \begin{array}{c} \fG'(\tilde{\xi}) \\[4pt] \fH'(\tilde{\xi}) \end{array} 
\right) d\tilde{\xi}
\end{equation}
%
\begin{equation}
1 = \fH^2 \fP'
\end{equation}
%
\begin{equation}
\left\{ \begin{array}{ccc}
\displaystyle \lim_{\xi\to\infty} \fH''(x) & = & \gamma \\
\displaystyle \lim_{\xi\to\infty} \fG'(x) & = & \gamma/2  
\end{array} \right.
\end{equation}
%
\begin{equation}
\fK = \lim_{\xi\to 0} 3 \sqrt{2\pi} \sqrt{\xi} \fH'(\xi)
\end{equation}
\\
Where $\gamma = M \left( \frac{36(1-\nu^2)^2}{\pi E^2 \mu c \ell^5} 
\right)^{1/3}$. 

\section{Relating the two scalings}
One can show that 
\[\gamma = 1/\lambda^{1/3} \]
\[\fG = \frac{1}{\lambda^{1/3}} \sG \qquad \fH = \frac{1}{\lambda^{1/3}} \sH\]
\[\fP = \frac{1}{\lambda^{1/3}} \sP \qquad \fK = \frac{1}{\lambda^{1/3}} \sK\]
The point of this, is that in earlier analysis (done elsewhere), we looked 
at $\sH$ near $\xi=0$. We found the asymptotic form
\[ \lambda = \lambda_0 + \cE(\sK)\lambda_1 + \dots \]
%
\[ \sH(\xi) = (A_0\xi^{2/3}+\dots) + \cE(\sK)\left(\frac{A_0 \lambda_1}
{3 \lambda_0} \xi^{2/3} + \xi^s + \dots \right) + \dots \]
%
\[ \sP(\xi) = (-\frac{3\lambda_0}{A_0^2}\xi^{-1/3}+\dots) + \cE(\sK)
\left(\frac{2\pi A_0 \lambda_1}{9 \lambda_0 \sqrt{3}} \xi^{-1/3} + 
\frac{4\pi}{9\sqrt{3}(1-s)}\xi^{s-1} + \dots \right) + \dots \]
%
\[ \cE(\sK) = C \sK^u \lambda_0^{2s-1}\]
The reason we've bothered with this other scaling ($\fH,\fK,\dots$), is 
that one notices that the equations near zero are almost exactly the same. 
The only difference is that $\lambda$ is set to 1. Since all of this is based 
near $\xi=0$, the boundary conditions at $\infty$ are not important. Thus we 
have that 
\[ \fH(\xi) = \left( \left(\frac{243}{4\pi^2}\right)^{1/6} \xi^{2/3}+\dots
\right) + \cE(\fK)\left( \xi^s + \dots \right) + \dots \]
%
\[ \fP(\xi) = \left(-\left( \frac{2\pi}{3} \right)^{2/3} 
\xi^{-1/3}+\dots \right) + \cE(\fK)
\left( \frac{4\pi}{9\sqrt{3}(1-s)}\xi^{s-1} + \dots \right) + \dots \]
%
\[ \cE(\fK) = C \fK^u\]
from which one can calculate $C$. The point is that it's the same $C$ in both
equations, which must be some numerical constant dependent on the near tip
geometry. One hopes that it is easier to calculate $C$ in the $\fH,\fK,\dots$
scaling.

%
%
% References/Bibliography ////////////////////////////////////////////////////
%
%\clearpage 
%\begin{thebibliography}{9}  
%
%\bibitem{Pedlosky}
%Pedlosky, J.,
%\emph{Geophysical Fluid Dynamics,}
%Springer-Verlag,
%1979.
%
%
%\end{thebibliography}
\end{document}

