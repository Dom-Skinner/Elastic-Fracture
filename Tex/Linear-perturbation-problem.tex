%
%
\documentclass{article}
\usepackage{amsmath}
\usepackage{graphicx}
\usepackage{color}
\usepackage{amsfonts}
\usepackage[margin=4cm]{geometry}
\newcommand{\cE}{\mathcal{E}}                               %
\newcommand{\cM}{\mathcal{M}}                               %
\newcommand{\cC}{\mathcal{C}}                               %
\newcommand{\cK}{\mathcal{K}}                               %
\newcommand{\bs}{\boldsymbol}                               %

\begin{document}

\title{Linear perturbation problem}
\author{Dominic Skinner}
\maketitle
Here we consider the linear perturbation problem, and how it can be solved
numerically. First we rescale into dimensionless parameters. Recall the full
equations are:
\begin{equation}
 \left( \begin{array}{c} p(z) \\ 0 \end{array} \right) =
\frac{E}{4\pi (1-\nu^2)} \int_0^{\infty} 
\underline{\underline{K}}(x-z) 
\left( \begin{array}{c} g'(x) \\ h'(x) \end{array} \right) dx
\end{equation}
\\
\begin{equation}
12\mu c = h^2 p'
\end{equation}
\\
\begin{equation}
\left\{ \begin{array}{ccc}
\displaystyle \lim_{x\to\infty} h''(x) & = & \frac{12(1-\nu^2)}{E\ell^3} M \\
\displaystyle \lim_{x\to\infty} g'(x) & = & \frac{6(1-\nu^2)}{E\ell^3} M 
\end{array} \right.
\end{equation}
\\
\begin{equation}
K_I = \lim_{x\to 0} \frac{E}{1-\nu^2} \sqrt{\frac{\pi}{8}} \sqrt{x} h'(x)
\end{equation}
\\
\section{Rescaling}
Let us use a length scale $\ell$, pressure scale $p^* = \frac{E}{12(1-\nu^2)}$,
and a time scale $t^* = 12\mu/p*$. We define the following dimensionless
parameters.
\[\cM = \frac{M}{p^* \ell^2}, \qquad \cC = \frac{c}{\ell/t^*} = \frac{12\mu c}
{p^* \ell}, \qquad \cK = \frac{K_I}{p^* \ell ^{1/2}} \]
We also define the variables (with $\alpha$ and $\beta$ dimensionless 
rescalings to be determined)
\[ x = \ell \xi, \qquad K_{ij} = \Lambda_{ij}/ \ell, \qquad
h = \alpha \ell H(\xi), \qquad p = \beta p^* \Pi ( \xi) \] 
So that
\[ \left( \begin{array}{c} \Pi \\ 0 \end{array} \right)
= \frac{3\alpha}{\pi\beta} \int \Lambda 
 \left( \begin{array}{c} G' \\ H' \end{array} \right) d\xi, \qquad
H^2 \Pi' = \frac{\cC}{\alpha^2 \beta}, \] \\
\[ H'' \to \cM/\alpha, \qquad
3\sqrt{2\pi\xi}H' \sim \frac{K_I}{(4\pi \mu x p^{*2} \ell^{1/2})^{1/3}} \]
\\
Choosing $\alpha = \pi \beta/3 = \cM$, $\displaystyle \lambda = 
\frac{\pi \cC}{3 \cM^3} = \frac{4\pi \mu c p^{*2} \ell^5}{M^3}$ gets 
Tim's scalings. 
\[ \left( \begin{array}{c} \Pi \\ 0 \end{array} \right)
= \int \Lambda 
 \left( \begin{array}{c} G' \\ H' \end{array} \right) d\xi, \qquad
H^2 \Pi' = \lambda, \] 
\[ H'' \to 1, \qquad
3\sqrt{2\pi\xi}H' \sim \frac{K_I}{M \ell^{-3/2}} \equiv \kappa \]
\\
Now suppose that $(G_0, H_0, \Pi_0, \lambda_0)$ gives the solution for 
$\kappa=0$. The outer limit of the LEFM solution is
\[ H \sim H_0 + \cE(\kappa) \left( \frac{\tilde{A} \lambda_1}{3\lambda_0^{2/3}}
\xi^{2/3} + \xi^s + \dots \right) \]
where $\cE = C \kappa^{4-6s} \lambda_0^{2s-1}$, $s = 0.138673$, and 
$C = \beta_1 (2/9\pi)^{2-3s}(1/4\pi)^{2-3s} = 8.99 \times 10^{-5}$.
Working to first order in $\cE$ we have the linear outer problem
\[ \left( \begin{array}{c} \Pi_1 \\ 0 \end{array} \right)
= \int \Lambda 
 \left( \begin{array}{c} G_1' \\ H_1' \end{array} \right) d\xi, \qquad
H_0^2 \Pi_1' + 2 H_0H_1 \Pi_0'= \lambda_1, \] 
\[ H_1'' \to 0, \qquad H_1 \sim \xi^s + \frac{\tilde{A} \lambda_1}{3 
\lambda_0^{2/3}} \xi^{2/3} \]
But we also have to zeroth order
\[ \left( \begin{array}{c} \Pi_0 \\ 0 \end{array} \right)
= \int \Lambda 
 \left( \begin{array}{c} G_0' \\ H_0' \end{array} \right) d\xi, \qquad
H_0^2 \Pi_0' = \lambda_0, \] 
\[ H_0'' \to 1, \qquad H_0 \sim \tilde{A} \lambda_0^{1/3} \xi^{2/3} \]
Subtracting a scaled version of the first order solution from the zeroth
order solution, we get that
\[ \left( \begin{array}{c} \Pi_0 - a \Pi_1 \\ 0 \end{array} \right)
= \int \Lambda 
 \left( \begin{array}{c} G_0'- aG_1' \\ H_0'-aH_1' \end{array} \right) d\xi, 
\quad H_0^2 (\Pi_0-a\Pi_1)'+2H_0(H_0 - a H_1)\Pi_0' = 3\lambda_0 - 
a \lambda_1, \] 
\[ (H_0-aH_1)'' \to 1, \qquad H_0 - a H_1 \sim 
\frac{\tilde{A}}{3 \lambda_0^{2/3}} \xi^{2/3}(3\lambda_0 - a\lambda_1) 
- a \xi^s\]
Setting $a = 3\lambda_0/\lambda_1$ and defining $\tilde{H} = 
H_0 - a H_1$ etc. gets the equations 
\[ \left( \begin{array}{c} \tilde{\Pi} \\ 0 \end{array} \right)
= \int \Lambda 
 \left( \begin{array}{c} \tilde{G}' \\ \tilde{H}' \end{array} \right) d\xi, 
\quad H_0^2 \tilde{\Pi}'+2H_0\tilde{H}\Pi_0' = 0, \] 
\[ \tilde{H}'' \to 1, \qquad \tilde{H} \sim - \frac{3\lambda_0}{\lambda_1} 
\xi^s \]
Note that this is a slightly different scaling than proposed before, but 
this has the same boundary conditions at $\infty$ as before, and
the elasticity integral equation is exactly the same as before. This means
the old code can hopefully be reused, as well as the fact that we can easily
impose conditions at $\infty$, whereas it is non-obvious how to easily impose 
them at 0.
\section{Numerical strategy}
The old method of linearising via $\tilde{H}'(\xi) \approx 
\frac{1}{\sqrt{\xi}}(a\xi+b)$ may no longer be too helpful, since we do not 
predict such a $\xi^{-1/2}$ singularity. It is possible (and even likely) 
that $\tilde{G}'$ still has such a singularity, but we predict $\tilde{H}'$ 
to have a $\xi^{s-1}$ singularity near the origin. Therefore, we shall use
the $\tilde{H}'(\xi) \approx \xi^{s-1}(a\xi+b)$ 
approximation.
\\
\\
Recall that,
\[ \left( \begin{array}{c} \tilde{\Pi}(z_1) \\ \vdots \\ \tilde{\Pi}(z_{n-1}) 
\\[4pt] 0 \\ \vdots \\ 0 \end{array} \right) =
\left( \begin{array}{ccc} B_{1,1} & \cdots & B_{1 , 2n} \\
\vdots & \ddots & \vdots \\ B_{2(n-1),1} & \cdots & B_{2(n-1) , 2n} 
\end{array}
\right) \bs{\gamma} = BT\bs{\theta} \]
Where $B$ is in lieu of the Kernel integral, $T$ is the interpolation matrix,
and recall that 
\[ \bs{\gamma} = (a_1, \dots, a_n, b_1, \dots , b_n, c_1, \dots, c_n, d_1, \dots 
 d_n) \]
\[ \bs{\theta} = (a_1\xi_1+b_1, \dots, a_n\xi_n+b_n, c_1\xi_1+d_1, \dots,
 c_n \xi_n +  d_n) \]
Now, the difference from before is that the second equation for $\tilde{\Pi}$ 
is linear in $\tilde{H}$. 
\[ \tilde{\Pi} = \int_z^{\infty} \frac{2 \tilde{H} \Pi_0'}{H_0} d\xi \]
So (after a struggle) one might be able to alternatively represent 
$\tilde{\Pi}$ via 
\[ \left( \begin{array}{c} \tilde{\Pi}(z_1) \\ \vdots \\ \tilde{\Pi}(z_{n-1}) 
\\[4pt] 0 \\ \vdots \\ 0 \end{array} \right) =
\left( \begin{array}{ccc} R_{1,1} & \cdots & R_{1 , 2n} \\
\vdots & \ddots & \vdots \\ R_{n-1,1} & \cdots & R_{n-1 , 2n} 
\\ 0 & \cdots & 0  \\
\vdots & \ddots & \vdots \\ 0 & \cdots & 0 
\end{array} \right) \bs{\theta} \]
To get that $(BT-R)\bs{\theta} = 0$. We can add another two rows to the
matrix by demanding $\theta_n=1/2$ and 
$\displaystyle \frac{\theta_{2n}-\theta_{2n-1}}{x_{2n}- x_{2n-1}}=1$.
This gives us a matrix equation to solve: $A \bs{\theta} = \bs{c}$, where
$\bs{c}=0$ except $c_n=1/2$, $c_{2n}=1$. Inverting $A$ should get the 
required answer, no iteration needed.
\section{Numerical Details}
\subsection{Lubrication matrix}
We need to figure out a way to work out the matrix $R$. Our best possible 
representation of $H_0$ and $\tilde{H}$ look something like
\[ H_0(\xi) = \left\{ \begin{array}{cc} \xi^{2/3}(w_i^0\xi+e_i^0)+r_i^0 &
i <t \\ w_i^0\xi^2+e_i^0\xi + r_i^0 & i \geq t \end{array} \right. \]
%
\[ \tilde{H}(\xi) = \left\{ \begin{array}{cc} \xi^{s}(w_i\xi+e_i)+r_i &
i <t \\ w_i\xi^2+e_i\xi + r_i & i \geq t \end{array} \right. \]
For $\xi \in [\xi_i,\xi_{i+1}]$. 

Recall the equation we are trying to solve:
\[ \tilde{\Pi} = \int_z^{\infty} \frac{2 \tilde{H} \Pi_0'}{H_0} d\xi \]
We want to solve for $(\tilde{\Pi}(z_1), \dots \tilde{\Pi}(z_{n-1}))$.
We will also use the lubrication equation $\Pi_0'H_0^2=\lambda_0$ 
to remove $\Pi_0'$.
\[ \tilde{\Pi}(z_k) = \int_{z_k}^{\xi_{k+1}} \frac{2\lambda_0 \tilde{H}}{H_0^3} 
d\xi + \sum_{r=k+1}^{n-1}\int_{\xi_r}^{\xi_{r+1}} \frac{2\lambda_0 \tilde{H}}
{H_0^3} d\xi + \int_{\xi_n}^{\infty} \frac{2\lambda_0 \tilde{H}}
{H_0^3} d\xi\]
It becomes a question of approximating integrals of the form
\[ \int_{\tau_1}^{\tau_2} \frac{2\lambda_0 \tilde{H}}{H_0^3} d\xi \]
Where $\tau_2 = \xi_{r+1}$, $\tau_1 = \xi_r$ or $z_r$. The way chosen to 
approximate the integrand was
\[ 2\lambda_0\tilde{H}/H_0^3 = 
\left\{ \begin{array}{cc} \xi^{s-4/3}(a_i\xi+b_i) &
i <t \\ \xi^{-5}(a_i\xi+b_i) & i \geq t \end{array} \right. \]
To reflect the suspected asymptotics of the integrand, near 0 and near 
$\infty$. The general way to think about this is if 
$ 2\lambda_0\tilde{H}/H_0^3 = \xi^{\alpha}(a_i\xi+b_i) $, then
\[ \int_{\tau_1}^{\tau_2} \frac{2\lambda_0 \tilde{H}}{H_0^3} d\xi =
a_i\left[ \frac{\tau_2^{\alpha+2}-\tau_1^{\alpha+2}}{\alpha+2}\right]+
b_i\left[ \frac{\tau_2^{\alpha+1}-\tau_1^{\alpha+1}}{\alpha+1}\right] \]
By matching at the endpoints, one can determine the coefficients, $a,b$
linearly in terms of the $w,e,r,$ coeffs. Further, since the integral is
clearly linear in $a,b$,  one can find formulae to make the integral linear
in the $w,e,r$. 
\subsection{Integral Kernel}
We wish to recalculate the integral kernel, given that near the origin,
$\tilde{H}' = \xi^{s-1} (a_i \xi + b_i)$. We assume the same $\tilde{G}'$
spacing as always. This raises the problem of calculating such integrals as
\[ \int_{\ell_\xi}^{u_\xi} \frac{48(\xi-z)^2 -64}{(\xi-z)( (\xi-z)^2+4)^3}
\xi^{\alpha} d\xi \]
\[ \int_{\ell_\xi}^{u_\xi} \frac{24(\xi-z)^2 -32}{( (\xi-z)^2+4)^3}
\xi^{\alpha} d\xi \]
Where $\alpha = s,s-1$ and $u_\xi = \xi_{k+1}$, $\ell_{\xi} = \xi_k$.
In a deviation from what was done previously, there is no nice analytic solution
to this equation in for general $\alpha$. There is a solution in terms of 
hypergeometric functions, but these take a long time to compute, and worse, 
don't seem to give a very good answer. That is to say that they don't capture
the nature of the singularity, removed by using a Cauchy principal value.
We can rewrite the first integral as 
\[ \int_{\ell_\xi}^{u_\xi} \frac{48(\xi-z)^2 -64}{(\xi-z)( (\xi-z)^2+4)^3}
\xi^{\alpha} d\xi = \int_{\ell_\xi}^{u_\xi} \frac{(\xi-z)^5 + 12 (\xi-z)^3 + 
96(\xi-z)}{((\xi-z)^2 +4)^3} \xi^{\alpha} d \xi - 
 \int_{\ell_\xi}^{u_\xi} \frac{\xi^{\alpha}}{\xi-z} d\xi \]
Which splits the integrals into singular and non singular parts. The non 
singular parts are mostly well behaved and can be integrated with matlabs
inbuilt routine. However, when $\alpha = s-1$, they have an integrable
singularity near the origin that numerical integrators can't handle. This 
can be removed manually via integration by parts. For instance, consider
a well behaved function $f$. Then
\[ \int_{u_\xi}^{\ell_\xi} f(\xi - z) \xi^{\alpha} d\xi = \left[
\frac{\xi^{\alpha+1}}{\alpha+1} f(\xi - z) \right]_{\ell_\xi}^{u_\xi} 
- \int_{\ell_\xi}^{u_\xi} \frac{\xi^{\alpha+1}}{\alpha+1} f'(\xi -z)d\xi\]
Where for $-1 < \xi < 0$ the left hand side is numerically suspect, whereas
the right hand side is numerically tractable.
\\
\\
This leaves the important issue of the singularity, and the Cauchy principal
value, ie. the integral
\[ \int_{\ell_\xi}^{u_\xi} \frac{\xi^{\alpha}}{\xi-z} d\xi \]
We will do some substutions and some reshuffling until this takes a form that
is convenient to work with. 
\[ \int_{\ell_\xi}^{u_\xi} \frac{\xi^{\alpha}}{\xi-z} d\xi = 
z^{\alpha} \int_{\ell_\xi/z}^{u_\xi/z} \frac{u^{\alpha}}{u-1} du = 
 z^{\alpha} \left\{ \int_{\ell_\xi/z-1}^{u_\xi/z-1} 
\frac{\theta^{\alpha}-1}{\theta} d\theta +  \int_{\ell_\xi/z-1}^{u_\xi/z-1}
1/\theta d \theta 
\right\}\]
Where $\frac{\theta^{\alpha}-1}{\theta}$ is a well behaved function near 0.
We also can show that 
\[\int_{\ell_\xi/z-1}^{u_\xi/z-1} 1/\theta \; d \theta = \log \left| 
\frac{u_\xi - z}{z - \ell_\xi} \right|\]
Which conveniently holds whether or not the interval contains the origin.
The algorithm goes something like:
\[ \int_{\ell_\xi}^{u_\xi} \frac{48(\xi-z)^2-64}{(\xi-z)( (\xi-z)^2 + 4)^3} 
\xi^{\alpha} d\xi= 
\underbrace{\int_{\ell_\xi}^{u_\xi} 
\frac{(\xi-z)((\xi-z)^4+12(\xi-z)^2-96)}{( (\xi-z)^2 + 4)^3}}_{I_1} 
\xi^{\alpha} d\xi +
\underbrace{ \int_{\ell_x}^{u_x} \frac{\xi^{\alpha}}{z-\xi} d\xi}_{I_2} \]

\subsubsection*{I$_1$}
Is $\ell_x < 0.5$ and $\alpha < 0$?
\begin{itemize}
\item Yes. The integral as it stands will have trouble at $0$ Write
\begin{align*}
 I_1 =& \left[ \frac{\xi^{\alpha+1}}{\alpha+1}
\frac{(\xi-z)((\xi-z)^4+12(\xi-z)^2-96)}{( (\xi-z)^2 + 4)^3}
\right]_{\ell_\xi}^{u_\xi} \\
& - \int_{\ell_x}^{u_x} \frac{\xi^{\alpha+1}}{\alpha+1} \frac{d}{dx}
\left\{ \frac{(\xi-z)((\xi-z)^4+12(\xi-z)^2-96)}{( (\xi-z)^2 + 4)^3}
\right\} d\xi \end{align*}
Where both parts of the expression are numerically tractable.
\item No. Just evaluate the integral numerically, as is.
\end{itemize}
\subsubsection*{I$_2$}
\begin{itemize}
\item Case: $\alpha > 0$ or $\ell_\xi/z > 1/2$. 
\\
\[ I_2 = - z^{\alpha} \left\{ \int_{\ell_\xi/z-1}^{u_\xi/z -1} 
\frac{(1+\theta)^{\alpha} -1}{\theta} d\theta + \log \left| 
\frac{u_\xi - z}{z - \ell_\xi} \right| \right\} \]
\item Case: $\alpha < 0$, $u_\xi/z < 1/2$. No problems with Cauchy PV,
but might have some issues with integrable singualarity at 0.
\[ I_2 = - z^{\alpha} \left\{ \left[ \frac{u^{\alpha+1}}{\alpha+1} 
\frac{1}{u-1} \right]_{\ell_\xi/z}^{u_\xi/z} + \int_{\ell_\xi/z}^{u_\xi/z}
\frac{u^{\alpha+1}}{\alpha+1} 
\frac{1}{(u-1)^2} du \right\}\]
\item Case: $\alpha < 0$,  $\; \ell_\xi/z < 1/2 < u_\xi/z$. Trouble from
both poles. We split the integral
\begin{align*} I_2 =& -z^{\alpha} \left\{ 
\int_{\ell_\xi/z}^{1/2} \frac{u^{\alpha}}{u-1} du  + 
\int_{1/2}^{u_\xi/z} \frac{u^{\alpha}}{u-1} du 
\right\}  \\
= & -z^{\alpha} \left\{ \left[ \frac{u^{\alpha+1}}{\alpha+1} 
\frac{1}{u-1} \right]_{\ell_\xi/z}^{1/2} + \int_{\ell_\xi/z}^{1/2}
\frac{u^{\alpha+1}}{\alpha+1} 
\frac{1}{(u-1)^2} du  \right. \\
& \qquad \qquad + \left. \int_{-1/2}^{u_\xi/z -1} 
\frac{(1+\theta)^{\alpha} -1}{\theta} d\theta + \log \left| 
2u_\xi/z - 1 \right| 
\right\} 
\end{align*}
\end{itemize}

%
%
% References/Bibliography /////////////////////////////////////////////////////
%
%\clearpage 
%\begin{thebibliography}{9}  
%
%\bibitem{Pedlosky}
%Pedlosky, J.,
%\emph{Geophysical Fluid Dynamics,}
%Springer-Verlag,
%1979.
%
%
%\end{thebibliography}
\end{document}

