%
%
\documentclass{article}
\usepackage{amsmath}
\usepackage{graphicx}
\usepackage{color}
\usepackage{caption}
\usepackage{amsfonts}
\usepackage[margin=4cm]{geometry}
\usepackage{tikz}
\newcommand{\cE}{\mathcal{E}}                               %

\begin{document}

\title{Rescaling the equations and boundary conditions}
\author{Dominic Skinner}
\maketitle
In this document, we ignore all of the physics and geometry of the problem,
to just consider the governing equations on their own. There are an abundance
of dimensional parameters, but we consider most of them to be fixed. 
What we are really interested in, is the relationship
between $M,K_I$ and $c$.
\\
\begin{equation}
 \left( \begin{array}{c} p(z) \\ 0 \end{array} \right) =
\frac{E}{4\pi (1-\nu^2)} \int_0^{\infty} 
\underline{\underline{K}}(x-z) 
\left( \begin{array}{c} g'(x) \\ h'(x) \end{array} \right) dx
\end{equation}
\\
\begin{equation}
12\mu c = h^2 p'
\end{equation}
\\
\begin{equation}
\left\{ \begin{array}{ccc}
\displaystyle \lim_{x\to\infty} h''(x) & = & \frac{12(1-\nu^2)}{E\ell^3} M \\
\displaystyle \lim_{x\to\infty} g'(x) & = & \frac{6(1-\nu^2)}{E\ell^3} M 
\end{array} \right.
\end{equation}
\\
\begin{equation}
K_I = \lim_{x\to 0} \frac{E}{1-\nu^2} \sqrt{\frac{\pi}{8}} \sqrt{x} h'(x)
\end{equation}
\\
We switch to working with dimensionless variables, 
\[ (x,p,h,g,\underline{\underline{K}}) \to (\xi, \Pi, H, G, 
\underline{\underline{\Lambda}} )\]
Let us do this by the following transforms:
\begin{align*}
x &= \ell \xi \\
p(x) &= \beta \Pi (\xi) \\
h(x) &= \alpha H(\xi) \\
g(x) &= \alpha G(\xi) \\
\underline{\underline{K}}(x) &= \frac{1}{\ell} 
\underline{\underline{\Lambda}}(\xi) \end{align*}
At this point we have already made some choices. I would claim the
choices made this far are ``natural'' although I don't have any more
justification than that. The equations become
\\
\begin{equation}
 \left( \begin{array}{c} \Pi(\xi) \\ 0 \end{array} \right) =
\frac{E\alpha}{4\pi (1-\nu^2)\beta \ell} \int_0^{\infty} 
\underline{\underline{\Lambda}}(\tilde{\xi}- \xi) 
\left( \begin{array}{c} G'(\tilde{\xi}) \\[4pt] H'(\tilde{\xi}) \end{array} 
\right) d\tilde{\xi}
\end{equation}
\\
\begin{equation}
\frac{12\mu c\ell}{\alpha^2 \beta}  = H^2 \Pi'
\end{equation}
\\
\begin{equation}
\left\{ \begin{array}{ccc}
\displaystyle \lim_{\xi\to\infty} H''(x) & = & \frac{12(1-\nu^2)}
{E\ell\alpha} M \\
\displaystyle \lim_{\xi\to\infty} G'(x) & = & \frac{6(1-\nu^2)}{E\ell\alpha} M 
\end{array} \right.
\end{equation}
\\
\begin{equation}
K_I = \lim_{\xi\to 0} \frac{E\alpha}{1-\nu^2} \sqrt{\frac{\pi}{8\ell}} 
\sqrt{\xi} H'(\xi)
\end{equation}
\\
We now have several choices. The first to be explored, is to take equations
5 and 6 and set the dimensionless parameters appearing in them to unity.
\section{Scaling out $c$}
Set $\displaystyle \frac{E\alpha}{4\pi (1-\nu^2)\beta \ell} =1$ 
and $\displaystyle \frac{12\mu c\ell}{\alpha^2 \beta}=1 $
which uniquely determines $\alpha, \beta$ as
\[ \alpha^3 = \frac{48\pi(1-\nu^2) \mu c \ell^2}{E} \qquad
 \beta^3 = \frac{3\mu c E^2}{\pi (1 - \nu^2) \ell} \]
We then have the relavent equations as 
\\
\begin{equation}
 \left( \begin{array}{c} \Pi(\xi) \\ 0 \end{array} \right) =
\int_0^{\infty} 
\underline{\underline{\Lambda}}(\tilde{\xi}- \xi) 
\left( \begin{array}{c} G'(\tilde{\xi}) \\[4pt] H'(\tilde{\xi}) \end{array} 
\right) d\tilde{\xi}
\end{equation}
\\
\begin{equation}
1 = H^2 \Pi'
\end{equation}
\\
\begin{equation}
\left\{ \begin{array}{ccc}
\displaystyle \lim_{\xi\to\infty} H''(x) & = & \gamma \\
\displaystyle \lim_{\xi\to\infty} G'(x) & = & \gamma/2  
\end{array} \right.
\end{equation}
\\
\begin{equation}
K_I = \lim_{\xi\to 0} \frac{E\alpha}{1-\nu^2} \sqrt{\frac{\pi}{8\ell}} 
\sqrt{\xi} H'(\xi)
\end{equation}
\\
Where $\gamma = M \left( \frac{36(1-\nu^2)^2}{\pi E^2 \mu c \ell^5} 
\right)^{1/3}$. Given equations 9,10,11, we can solve for $K_I$. It is 
clear from the form of them, that the only way the physical parameters
can enter the solution is through $\gamma$. Thus, the relationship between
$K_I$ and the other physical quantities \emph{must} be of the form
\[ K_I = E^{2/3} \mu^{1/3} c^{1/3} \ell^{1/6}(1-\nu^2)^{-2/3} f(\gamma) \]
N.B. if all we are interested in is $K_I,c,M$, get the relationship
\[K_I = c^{1/3} \tilde{f}(M/c^{1/3}) \]
but this does depend on other physical parameters.
\section{Scaling out $\gamma$}
Note that this is the scaling that is used in other documents, and Tim's
work.
Set $\displaystyle \frac{E\alpha}{4\pi (1-\nu^2)\beta \ell} =1$ 
and $\displaystyle \frac{12(1-\nu^2)}{E \ell \alpha}M=1 $. We get that
\[ \alpha = \frac{12(1-\nu^2)M}{E\ell} \qquad \beta = \frac{3M}{\pi \ell^2} \]
We then have the relevant equations as 
\\
\begin{equation}
 \left( \begin{array}{c} \Pi(\xi) \\ 0 \end{array} \right) =
\int_0^{\infty} 
\underline{\underline{\Lambda}}(\tilde{\xi}- \xi) 
\left( \begin{array}{c} G'(\tilde{\xi}) \\[4pt] H'(\tilde{\xi}) \end{array} 
\right) d\tilde{\xi}
\end{equation}
\\
\begin{equation}
\lambda = H^2 \Pi'
\end{equation}
\\
\begin{equation}
\left\{ \begin{array}{ccc}
\displaystyle \lim_{\xi\to\infty} H''(x) & = & 1 \\
\displaystyle \lim_{\xi\to\infty} G'(x) & = & 1/2  
\end{array} \right.
\end{equation}
\\
\begin{equation}
K_I = \lim_{\xi\to 0} 3M \sqrt{\frac{2\pi}{\ell^3}} 
\sqrt{\xi} H'(\xi)
\end{equation}
\\
Where $\lambda = \frac{\pi \mu c E^2 \ell^5}{36(1-\nu^2)^2 M^3}$. 
Thus $K_I$ must be of the form
\[K_I = M\ell^{-3/2} s(\lambda) \]
for some function $s$. This is not a new relationship, writing 
$s = \lambda^{1/3} \tilde{s}(\lambda)$ recovers the exact same relationship
that we had earlier.
We can also rescale $K_I$ via $K_I = M\ell^{-3/2} \kappa$, so that 
$\displaystyle \kappa = \lim_{\xi \to 0} 3\sqrt{2\pi} \sqrt{\xi}H'(\xi) $
and so $\kappa = \kappa(\lambda)=s(\lambda)$.
\\
\\
Note that the value of $\lambda_0$ s.t. $\kappa(\lambda_0)=0$ does not depend
on any physical parameters. It can be found numerically; 
$\lambda_0 \approx 0.06$. For the small toughness solution, we have
\[\kappa(\lambda) \propto (\lambda - \lambda_0)^{1/u}\]
where $u\approx 3.17\dots$ is a number obtained by solving a transcendental
equation. Thus, for $K_I$ small we have the relationship
\[ \lambda = \lambda_0 + A(\ell^{3/2} K_I / M)^u \]
Where $A$ is a dimensionless constant that can be determined numerically
and does not depend on any physical parameters, $A \approx -0.21$. From this
we can calculate the dependance of $c$ on the other physical parameters, for 
the small toughness solution.
\[ c = \frac{36(1-\nu^2)^2M^3}{\pi \mu E^2 \ell^5}\left(\lambda_0 + A
(\ell^{3/2}K_I/M)^u \right) \]
%
%
% References/Bibliography //////////////////////////////////////////////\\/////////////
%
%\clearpage 
%\begin{thebibliography}{9}  
%
%\bibitem{Pedlosky}
%Pedlosky, J.,
%\emph{Geophysical Fluid Dynamics,}
%Springer-Verlag,
%1979.
%
%
%\end{thebibliography}
\end{document}

