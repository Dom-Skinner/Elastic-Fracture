%
%
\documentclass{article}
\usepackage{amsmath}
\usepackage{graphicx}
\usepackage{caption}
\usepackage{mathrsfs}
\usepackage{eufrak}
\usepackage{amsfonts}
\usepackage[margin=4cm]{geometry}
\newcommand{\cE}{\mathcal{E}}                
\newcommand{\fP}{\mathfrak{P}}               
\newcommand{\fH}{\mathfrak{H}}               
\newcommand{\fG}{\mathfrak{G}}               
\newcommand{\fK}{\mathfrak{K}}               
\newcommand{\sP}{\mathscr{P}}                
\newcommand{\sH}{\mathscr{H}}                
\newcommand{\sG}{\mathscr{G}}                
\newcommand{\sK}{\mathscr{K}}                

\begin{document}

\title{Numerically calculating dimensionless coefficients}
\author{Dominic Skinner}
\maketitle
Recall that
\[\gamma = 1/\lambda^{1/3} \]
\[\fG = \frac{1}{\lambda^{1/3}} \sG \qquad \fH = \frac{1}{\lambda^{1/3}} \sH\]
\[\fP = \frac{1}{\lambda^{1/3}} \sP \qquad \fK = \frac{1}{\lambda^{1/3}} \sK\]
The relevant asymptotic form is
\[ \lambda = \lambda_0 + \cE(\sK)\lambda_1 + \dots \]
%
\[ \sH(\xi) = (A_0\xi^{2/3}+\dots) + \cE(\sK)\left(\frac{A_0 \lambda_1}
{3 \lambda_0} \xi^{2/3} + \xi^s + \dots \right) + \dots \]
%
\[ \sP(\xi) = (-\frac{3\lambda_0}{A_0^2}\xi^{-1/3}+\dots) + \cE(\sK)
\left(\frac{2\pi A_0 \lambda_1}{9 \lambda_0 \sqrt{3}} \xi^{-1/3} + 
\frac{4\pi}{9\sqrt{3}(1-s)}\xi^{s-1} + \dots \right) + \dots \]
%
\[ \cE(\sK) = C \sK^u \lambda_0^{2s-1}\]
Or, in the alternative scaling
\[ \fH(\xi) = \left( \left(\frac{243}{4\pi^2}\right)^{1/6} \xi^{2/3}+\dots
\right) + \cE(\fK)\left( \xi^s + \dots \right) + \dots \]
%
\[ \fP(\xi) = \left(-\left( \frac{2\pi}{3} \right)^{2/3} 
\xi^{-1/3}+\dots \right) + \cE(\fK)
\left( \frac{4\pi}{9\sqrt{3}(1-s)}\xi^{s-1} + \dots \right) + \dots \]
%
\[ \cE(\fK) = C \fK^u\]
We wish to calculate the parameters $\lambda_0, \lambda_1, C$. To do this, we
will also introduce the parameter $D$,
\[ D = C \lambda_1 \lambda_0^{2s-1} \]
so that $\lambda = \lambda_0 + D \sK ^u+ \dots$.
%
\subsection*{The good news}
Very rough estimates give that $\lambda_0 \approx 0.06$, $D \approx - 0.01$,
$\lambda_1 \approx -0.2$.
For values of $\sK < 1$, this gives $|\cE(\sK)| < 0.05$. Further, for 
$\sK < 0.5$ (we can go this low numerically, but not much further)
$|\cE(\sK)| < 0.005$. So for moderately low values of $\sK$, ignoring 
$\cE^2$ terms is an excellent approximation.
\\
\\
Plotting $\lambda$ against $\sK^u$ yields a very good approximation to a 
straight line. Using simple linear regression we can calculate $\lambda_0$,
and $D$, to around 2.s.f. accuracy.
\[ \lambda_0 \approx 0.0059 \]
\[ D \approx - 0.0074 \]
Given this value of $D$, we need only calculate $C$, to have the full range
of constants.
%
\subsection*{The bad news}
Let us move into the alternative scalings.
\[ \fH(\xi) = \fH_0(\xi) + \cE(\fK)\fH_1(\xi) \]
\[ \fP(\xi) = \fP_0(\xi) + \cE(\fK)\fP_1(\xi) \]
\[ \cE(\fK) = C \fK^u\]
Where this holds away from the LEFM boundary layer. We know that
$\fH_1(\xi) \to \xi^{s}$ near $\xi = 0$. Something we can do is to fix a point
$\xi$, and consider $\fH(\xi ; \fK^u)$. As $\fK^u \to 0$, plotting 
$\fH(\xi ; \fK^u)$ against $\fK^u$ yields a straight line. 
From this line, one calculates $\fH_0(\xi)$ and $C\fH_1(\xi)$.
\\
\\
The suspected cause of the problem is that $\fH_1 = \xi^s$ is never that
good of an approximation. In particular, plotting $C\fH_1$ against 
$\xi^s$ should yield a straight line through the origin with gradient $C$,
at least for small $\xi$. The problem is that it doesn't. I have not yet
resolved this issue. 
\\
\\
An alternative means of progress, would be to recall that $C$ arises
as the coefficient of an asymptotic expansion. Recall the problem:
\[ \Pi(\zeta) = -\int_0^{\infty}\frac{\eta'(\xi)}{\xi - \zeta} d\xi, \qquad
\eta^2 \Pi'=1, \qquad \eta(\xi) \to \frac{2}{3\sqrt{2\pi}}\xi^{1/2} \]
The solution to this problem has the far field asymptotics,
\[ \eta(\xi) = \left(\frac{243}{4\pi^2}\right)^{1/6} \xi^{2/3} 
+ C \xi^s + \dots \]
This may well turn out to be a much more convenient way to find $C$, as
the problem is certainly simpler.
%
%
% References/Bibliography ////////////////////////////////////////////////////
%
%\clearpage 
%\begin{thebibliography}{9}  
%
%\bibitem{Pedlosky}
%Pedlosky, J.,
%\emph{Geophysical Fluid Dynamics,}
%Springer-Verlag,
%1979.
%
%
%\end{thebibliography}
\end{document}

