Takes a value of $\lambda$ and returns the corresponding $K$ value.
Now using the ``scaled'' version, so the values of $P,M$ are implicitly
assumed to be $0,1$ respectively.

\textcolor{red}{Somewhat concerningly, $\bs{h}'$ is assumed to 
already have this spacing, which could potentially cause issues.
If you wanted to change the spacing you would have to do it in
two different places.}

Subroutines then return the kernel matrix \& the interpolate matrix.
The kernel matrix is in lieu of $\displaystyle \left( \begin{array}{c}
p \\ 0 \end{array} \right) = \int \underline{\underline{K}} \left(
\begin{array}{c} g' \\ h' \end{array} \right) $.
The interpolate matrix actually only appears here to add in boundary
conditions to the matrix $A$.

The matrix $A$ is set up, which is part kernel, part interpolate matrix,
same matrix as described earlier.

The rcond statement is testing how conditioned the matrix $A$ is, or how 
ameniable it is to being numerically inverted. There does not seem to be
a huge amount of point in adding this step in.

Then the iteration loop begins. Follows Newton's method for the equation
$f(\bs{h}') = A \bs{h}'$ and iterates via $\displaystyle \bs{h}'_{new} =
\bs{h}'_{old} + (A-Df|_{\bs{h}_{old}'})^{-1}(f(\bs{h}_{old}') - A\bs{h}_{old}')$
Where we already know $A$. $f, Df$ are provided via \texttt{hprime\_to\_p}
and $f, Df$ are called $p, dp$ respectively in the program.
